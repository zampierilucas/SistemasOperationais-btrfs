\documentclass[12pt]{article}

\usepackage{sbc-template}

\usepackage{graphicx,url}

%\usepackage[brazil]{babel}
\usepackage[utf8]{inputenc}


\sloppy

\title{Introduction to Brtfs}

\author{Giovanni Crepaldi\inst{1}, Giovanni Crepaldi\inst{2}, Lucas Marcon Zampieri\inst{3}}

\address{Universidade do Vale do Rio dos Sinos (Unisinos) \\
Porto Alegre -- RS -- Brazil
\nextinstitute
  Ciência da Computação -- Unisinos\\
  Porto Alegre, BR.
\nextinstitute
  Jogos Digitais -- Unisinos\\
  Porto Alegre, BR.
}

\begin{document}

\maketitle

\begin{abstract}

\end{abstract}

\begin{resumo}

\end{resumo}

\section{Informações gerais} \label{sec:firstpage}

\section{Estrutura de sistema de arquivos}

\section{Snapshots}

\section{Vantagens e desvantagens}

\section{Figuras e diagramas}\label{sec:figs}

\begin{thebibliography}{9}
\bibitem{Wikipedia}
Btrfs on wikipedia.
\\\texttt{https://pt.wikipedia.org/wiki/Btrfs}

\bibitem{linuxfoundation}
Linux Foundation - Introduction to btrfs.
\\\texttt{https://training.linuxfoundation.org/resources/\
webinars/introduction-to-btrfs/}

\bibitem{UnixUniverse}
UnixUniverse.
\\\texttt{https://unixuniverse.com.br/linux}

\bibitem{linuxFoundation2}
Linux foundation.
\\\texttt{https://www.linuxfoundation.org/blog/2010/11/\
new-btrfs-free-tutorial-and-request-for-end-user-feedback-on-btrfs/}

\bibitem{qwe}
qwe
\\\texttt{https://pt.qwe.wiki/wiki/Btrfs}

\end{thebibliography}

\end{document}

