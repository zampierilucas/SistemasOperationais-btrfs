\documentclass[12pt]{article}
\usepackage{sbc-template}
\usepackage{authblk}

\usepackage{graphicx,url}

%\usepackage[brazil]{babel}   
\usepackage[utf8]{inputenc}  


\sloppy

\title{Introduction to Brtfs}

\author[1]{Giovanni Crepaldi}
\author[2]{Lucas Marcon Zampieri}
\author[3]{Mauricio Sscherrer}
\affil[1]{giovannecrepaldi@gmail.com}
\affil[2]{zampierilucas@unisinos.br}
\affil[3]{mauricio.sscherer@gmail.com} 
\address{Universidade do Vale do Rio dos Sinos (Unisinos) \\
Porto Alegre -- RS -- Brazil
\nextinstitute
  Ciência da Computação -- Unisinos\\
  Porto Alegre, BR.
\nextinstitute
  Jogos Digitais -- Unisinos\\
  Porto Alegre, BR.
}

\begin{document} 

\maketitle

\section{Informações gerais} \label{sec:firstpage}

\section{Estrutura de sistema de arquivos}

O sistema de arquivos btrs é consistido por árvores encadeadas. Árvores são separadas por nodos, que são ordenados em uma árvore auto-balanceada, do tipo b-tree\cite{B-tree}. Nodos internos possuem referências a outros nodos internos do nível superior.

O sistema de arquivos btrfs não faz distinção de disco físico e lógico, sendo assim um endereço lógico pode referenciar um disco assim como N discos, de acordo com as configurações de RAID. Seus endereços lógicos antes de serem usados precisam ser convertidos para físicos, para isso existe um árvore chamada chunk-tree\cite{Btrfs-design}, Seu caminho reverso, isto é físico para lógico pode  ser convertido utilizando outra árvore chamada dev-tre.

O btrfs utiliza um sistema de escrita chamado de cópia em gravação(COW), em que um mesmo recurso é compartilhado entre diversos processos, até que seja modificado. Oque isto quer dizer é que diversos processos possuem um ponteiro para o mesmo endereço de processo, e o compartilham, até que um deles requisite um modificação, então é feita uma cópia privada do recurso para aquele processo, assim não afetando o dado que é compartilhado com outros processos. 
 
\section{Snapshots}

\section{Vantagens e desvantagens}

\section{Figuras e diagramas}\label{sec:figs}

\begin{thebibliography}{9}

\bibitem{Wikipedia}
Btrfs on wikipedia.
\\\texttt{https://pt.wikipedia.org/wiki/Btrfs}

\bibitem{linuxfoundation}
Linux Foundation - Introduction to btrfs.
\\\texttt{https://training.linuxfoundation.org/resources/\
webinars/introduction-to-btrfs/}

\bibitem{B-tree}
Rodeh, Ohad (2007). B-trees, shadowing, and clones (PDF). 
\\\texttt{https://www.usenix.org/legacy/events/lsf07/tech/rodeh.pdf}

\bibitem{Btrfs-design}
Wiki Kernel. Btrfs design.
\\\texttt{https://btrfs.wiki.kernel.org/index.php/Btrfs_design}

\bibitem{UnixUniverse}
UnixUniverse.
\\\texttt{https://unixuniverse.com.br/linux}

\bibitem{linuxFoundation2}
Linux foundation.
\\\texttt{https://www.linuxfoundation.org/blog/2010/11/\
new-btrfs-free-tutorial-and-request-for-end-user-feedback-on-btrfs/}

\end{thebibliography}

\end{document}
